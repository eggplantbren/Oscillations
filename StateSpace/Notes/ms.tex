\documentclass[letterpaper, 11pt]{article}
\usepackage{graphicx}
\usepackage{natbib}
\usepackage{amsmath}
\usepackage[left=3cm,top=3cm,right=3cm]{geometry}

\renewcommand{\topfraction}{0.85}
\renewcommand{\textfraction}{0.1}
\parindent=0cm
\newcommand{\btheta}{\boldsymbol{\theta}}

\title{Oscillations}
\author{Brendon J. Brewer}

\begin{document}
\maketitle

\section{Introduction}
The data are noisy measurements $\{Y_1, ..., Y_N\}$ at times
$\{t_1, ..., t_N\}$ with noise standard deviations (``error bars'')
$\{\sigma_1, ..., \sigma_N\}$. There is an underlying dynamics which is
given by the sum of $M$ damped, stochastically excited signals (modes)
$\{y_j(t)\}$:
\begin{eqnarray}
Y_i = \left(\sum_{j=1}^M y_j(t)\right) + \epsilon_i
\end{eqnarray}
where $\epsilon_i \sim \mathcal{N}(0, \sigma_i^2)$.
Each of the $M$ signals satisfies the SDE
\begin{eqnarray}
dy_j &=& v_j dt\\
dv_j &=& -\left[(2\pi\nu_j)^2y_j - \tau^{-1}v_j\right]dt + \beta dW
\end{eqnarray}
where $W(t)$ is white noise.

\end{document}

