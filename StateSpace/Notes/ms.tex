\documentclass[letterpaper, 11pt]{article}
\usepackage{graphicx}
\usepackage{natbib}
\usepackage{amsmath}
\usepackage{dsfont}
\usepackage[left=3cm,top=3cm,right=3cm]{geometry}

\renewcommand{\topfraction}{0.85}
\renewcommand{\textfraction}{0.1}
\parindent=0cm
\newcommand{\btheta}{\boldsymbol{\theta}}
\newcommand{\E}{\mathds{E}}

\title{Oscillations}
\author{Brendon J. Brewer}

\begin{document}
\maketitle

\section{Introduction}
{\bf Apologies if anyone who knows this stuff is reading and I make any basic
blunders. I'm just muddling my way through and am not an expert of this stuff.}\\

The data are noisy measurements $\{Y_1, ..., Y_N\}$ at times
$\{t_1, ..., t_N\}$ with noise standard deviations (``error bars'')
$\{\sigma_1, ..., \sigma_N\}$. There is an underlying dynamics which is
given by the sum of $M$ damped, stochastically excited signals (modes)
$\{y_j(t)\}$:
\begin{eqnarray}
Y_i = \left(\sum_{j=1}^M y_j(t)\right) + \epsilon_i
\end{eqnarray}
where $\epsilon_i \sim \mathcal{N}(0, \sigma_i^2)$.
Each of the $M$ signals satisfies the SDE
\begin{eqnarray}
dy_j &=& v_j \,dt\label{eq:sde1}\\
dv_j &=& -\left[\omega_j^2y_j - \tau^{-1}v_j\right] \,dt + \beta \,dW_j\label{eq:sde2}
\end{eqnarray}
where $W_j(t)$ has a white noise prior probability distribution.

I think what we need is the Fokker-Planck equation for $Y(t)$, and its
solutions. Let the probability density for $Y(t)$ at time $t$ be
$\rho(Y; t)$. We need to figure out what Equations~\ref{eq:sde1}
and~\ref{eq:sde2} imply about the evolution of $\rho(Y; t)$.
To first order, we have
\begin{eqnarray}
Y(t+h) &=& Y(t) + h\sum_{i=1}^M v_j(t) + \sigma(t)\,dW_{\rm noise}\\
y_j(t + h) &=& y_j(t) + h v_j(t)\\
v_j(t + h) &=& v_j(t) - h\omega_j^2y_j(t) - h\tau^{-1}v_j(t) + \beta \,dW_j
\end{eqnarray}
Since these are all linear, I think we can deduce that a joint PDF for
everything at time $t$ that is gaussian will remain so at time $t+h$.

Taking the expectation of everything (conditional on $Y(t)$), we get
\begin{eqnarray}
\E\left[Y(t+h)\right] &=& Y(t) + h\sum_{i=1}^M \E\left[v_j(t)\right]\\
\E\left[y_j(t + h)\right] &=& \E\left[y_j(t)\right] + h\E\left[v_j(t)\right]\\
\E\left[v_j(t + h)\right] &=& \E\left[v_j(t)\right] - h\omega_j^2\E\left[y_j(t)\right] - h\tau^{-1}\E\left[v_j(t)\right]
\end{eqnarray}



\end{document}

